\documentclass[10pt]{article}

\usepackage[margin=1in]{geometry}

\usepackage{enumitem}
\setlist{nolistsep}

\title{\vspace{-4em}6.S078 Update} \author{Troy Astorino \and Turner Bohlen \and
Craig Cheney \and Gus Downs}

\begin{document}
\maketitle
\vspace{-4em}

\section{Plan Progress} After meeting with Professor Gifford last week, we
started looking into 3D scanning with serious intent to pursue it. We found that
while there are a number 3D scanning products out there, no products target a
consumer market. We also liked that the same general technology could be applied
to many different market segments, keeping open the possibility of pivoting
without having to change the core technology. We discussed the possibilities and
opportunities of pursuing 3D scanning as a team, and made a final decision that
this is what we will focus on for this class.

We are planning to build a low-cost 3D scanner aimed at consumers, riding
alongside the current excitement in consumer 3D printing. Our current vision for
the target customer is a digital fabrication hobbyist/enthusiast/early-adopter,
or even small-scale manufacturers (machine shops). Our main design goals are to
bring resolution beyond the limit of typical machining tolerances (0.1 mm) while
keeping the unit affordable (below \$500). 

The plan moving forward is to continue to do market research, as well as
research the currently existing 3D scanning methods. By the end of this week
(2/22) we would like to have selected a few preliminary design schemes so that
we can start prototyping.

\section{Prototype Progress} As we have yet to select what approach we are
taking to 3D scanning, we have not started building prototype software or
hardware. We plan to start making progress on a prototype by the end of this
week.

\section{Baffling Variables}
We have many, many variables facing us at the moment, but none of them stand out
as more baffling than any of the others.

\section{Seven Day Plan} 
\begin{itemize} 
\item Continue to research methods used for 3D scanning
\item Obtain access to the Stellar materials for 6.838 (Advanced
  topics in computer graphics: Computational Fabrication) as well as meet with the
  class's professor, Wojciech Matusik. 
\item Craig is going to make contact with his ex-employer (AutoDesk) to research
  what is required for 3D scanning in consumer/industrial CAD. 
\item Do some image reconstruction into 3D models as one means of researching
  the technology
\item Design some potential scanner configurations
\item Contact early purchasers of 3D printers to gague
  interest/requirements/their use cases for a 3D scanner
\item Obtain some funding for prototyping
\end{itemize}

\section{People to Meet} We haven't identified anyone in particular we would
like to meet at this stage (other than the professor for 6.838 listed
above...we'll let you know if we have difficulty getting through to him).

\section{Desired Resources} As we are hardware startup, in
order to start prototyping we need some capital to purchase components. We
have access to all of the facilities that we will need, but at present we lack
the funding to purchase materials. In our meeting with Professor Gifford he
mentioned that it would be possible to get some form of funding, such as a
forgivable loan, from the course VC partners. We would like to pursue that possibility.

\end{document}