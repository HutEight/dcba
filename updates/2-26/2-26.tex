\documentclass[10pt]{article}

% small margins
\usepackage[margin=1in]{geometry}

% no large spaces in lists
\usepackage{enumitem}
\setlist{nolistsep}

\title{\vspace{-4em}6.S078 Update}
\author{Troy Astorino \and Turner Bohlen \and Craig Cheney \and Gus Downs}
\date{February 26, 2013}

\usepackage[colorlinks=true]{hyperref}

\begin{document}
\maketitle
\vspace{-4em}

\section{Plan Progress}
This past week has been filled with market and technology research. There is a
large existing market in 3D scanning, with market revenue over \$350 million in
2011. Automotive and aerospace are the largest end-user industries, followed by
machine shops, metal fabrication, and heavy machinery. Other end-user industries
include archaeology/museum curation, architecture, medicine, power generation,
animation/special effects, consumer, and education. According to some analysis
we found, the largest challenges facing the industry are high prices and complex
user interfaces. This falls in line with our own perceptions about the industry.
We also found that while it is typical for a large manufacturer to use 3D
scanners for production verification, the high cost of 3D scanners prevents
smaller manufactures from using them to the same end. We believe this is a
targetable market, and to address this market we need to be precise enough to
measure threading and bore diameter. For the growing consumer market, there are
free or low-cost applications based off image reconstruction. Autodesk's
\href{http://www.123dapp.com/catch}{123D Catch} is a popular free app, and there
are systems based off using the Microsoft Kinect. There are limitations to the
types of objects these kinds of systems can be used on (not effective on shiny,
regularly-shaped objects, or when there is a regular background), and at the
very least they require the user to move around the object and capture images,
taking a significant amount of time. The resolution of their scans can also be
lacking. Because these products are free, a consumer-targeted 3D scanner would
need to be sure differentiate itself on some levels. We think resolution,
variety of scannable objects, and level of automation are 3 factors a product
could be differentiated by.

In terms of technology research, we started by looking at all the technologies
currently used in 3D scanning, including computed tomography, opacity, contact
probes, triangulation, structured light, time of flight, and image
reconstruction. Computed tomography involves the use of x-rays, eliminating it
as a low cost possibility. Using x-rays to scan interior parts of objects would
be an interesting possibility, but we think for us it is best not to take that
path. There are also interesting approaches possibly for time-of-flight scans,
but the scan-speed to resolution trade-off that comes with time-of-flight is
undesirable. Contact probes are effective, but slow and expensive, and we don't
have any ideas to lower costs. We have decided that structured light is the most
promising technology for us to pursue. Falling costs in digital projection and
digital photography has recently lead to an increased academic attention on this
approach, and there are even relatively straightforward ways for dedicated
hobbyists to set up their own structured light 3D scanners. We think that
combining methods intelligently (e.g. using image reconstruction alongside
structured light scanning) could allow us to increase resolution, increase
effectiveness on highly specular objects, and keep costs low. If done well, this
would also allow us to push complexity into software and use cheap hardware.

In summary:
\begin{itemize}
\item We have narrowed our potential customers down to two markets: small
  manufacturers and consumers/hobbyists. Initial prototyping will be
  similar for either of these market segments, though for small manufacturers we
  need to focus more on scan accuracy of machining features, while for
  consumers/hobbyists we need to focus more strongly on simplicity of use.

\item Structured light is the principle technology we plan to use for scanning.
  We think we can combine structured light with image reconstruction and stero-vision
  methods to increase accuracy.
\end{itemize}

\section{Prototype Progress}
We still do not have prototype progress, but we have build time scheduled this
week.

\section{Baffling Variables}
We still have a large amount of uncertainty about the sizes of our potential
markets, and the customer requirements for these markets. We have scheduled
meetings that will help us answer our questions in these areas.

\section{Seven Day Plan}
\begin{itemize}
\item We have several meetings that scheduling has carried over from
  last week, so we are moving forward on those.
\item On Wednesday we are moving into our new workspace, where we can start
  building and prototyping.
\item Craig and Gus putting together a rough CAD model for a modular scanner
  composed of interlocking panels. 
\item Troy and Turner are putting together a basic structured light 3D scanner
  using a projector and a webcam
\item Troy and Turner are pushing our prototyping funding search out to more
  places 
\end{itemize}

\section{People to Meet}
We would like to meet with the course's VC advisors in order to try and get some
prototyping funding

\section{Desired Resources}
Same as above section: we would like to get a hold of some funding for
prototyping supplies.

\end{document}