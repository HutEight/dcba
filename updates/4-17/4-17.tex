\documentclass[10pt]{article}

% small margins
\usepackage[margin=1in]{geometry}

% no large spaces in lists
\usepackage{enumitem}
\setlist{nolistsep}

\usepackage[colorlinks=true]{hyperref}

\title{\vspace{-4em}6.S078 Update}
\author{Troy Astorino \and Turner Bohlen \and Craig Cheney \and Gus Downs}

\begin{document}
\maketitle
\vspace{-4em}

\section{Plan Progress}
Our main focus is currently still prototyping.  More details on this progress is
in the following section.

We have begun searching for an additional programmer to join the team. We had
originally planned to wait until we would hear back from accelerator programs,
but we judged this would take too long. This additional teammate could work with
us for only the summer, or could join as a cofounder.

In terms of market, another competitor joined the space recently. The \href{http://www.indiegogo.com/projects/photon-3d-scanner}{Photon 3D
Scanner} was launched a few
days ago on indiegogo. Their unit price is \$450. They are claiming .2 mm
accuracy for scanning a 4 in figurine. Their target customer is very close to
ours. They are aiming at shipping before the end of the year. They are the most
direct competition in our space so far, and have a very tangible product.  It
makes us think it may be valuable to focus on some differentiation, perhaps by
ensuring that our maximum scan size is larger than the alternatives.

We've outlined then the 5 year projection, but we stuck on trying to make sales
estimates. We tried to get some sales numbers from NextEngine, who is the
incumbent in this space, but they understandably did not want to give any
information about their unit sales and growth. Through a phone conversation,
they said that they believe they have a very competitive market position with no
real competitors (they view the import aspects for a product to be scan
accuracy, cost, and speed of use). Troy is meeting with one of his economics
professors to discuss techniques for approximating demand using data from a
different market segment (industrial 3D scanners) and a pseudo-complementary
good/analog good (3D printers).  Otherwise, we have made assumptions on unit
costs and development costs, and projecting a launch of a second product that
would be aimed at the larger industrial players.  This second product would
greatly increase the max scan size from the first product, and would severely
undercut the cost of the other industrial 3D scanners.  We also assumed
beginnings of research and development for 3D scanners that would scan the
interior of an object using x-rays.

\section{Prototype Progress}
The hardware prototyping phase is about to come into full swing. The CMOS sensor arrays and two types of 8051 microcontrollers have been purchased and received. Craig is about to start his 6.115 (Microcontroller Laboratory) final project, which is building the embedded system for the scanner, focusing mostly on the camera system. The data will be collected off the cameras and stored on an SD card, which can then be transferred to the computer which will do the object reconstruction. In the future this will be done through a USB interface, and no SD card.

Craig is working with two different cameras at this point. One is an OV3640, which is a CMOS sensor array. However surface mounting is proving to be difficult, as the chip uses an uncommon 56 pin Ball Grid Array (BGA) scheme that is not compatible with any off the shelf adapters for bread boards. This may mean creating a custom break out board, which is less than ideal for prototyping. 

Alternatively we have purchased a lower quality camera from SparkFun, which will serve as a backup in case the OV3640 proves too difficult to quickly mount. The contacts for this camera are easily solderable by hand. The downside of these cameras is that the documentation is not nearly as good as the OmniVision camera. There are blogs of people who have used the SparkFun cameras before. 

Once the mounting hurdle has been crossed, the next big challenge will be using
the correct transfer protocol to talk to the camera. The SparkFun camera uses an
I2C protocol, which is much more common than the OmniVision proprietary protocol
that the OV3640. 

Gus's first experiments with gratings and LEDs, and his subsequent calculations, showed
that we will need a lens optics system with basic LEDs, or will have to go to a
coherent light source. He is ordering parts for a next order analysis.

Software prototyping is continuing, fixing some portions of the scan progress
and developing the meshing system.  Soon we will be building interactions with
the hardware system.

\section{Baffling Variables}
We are now confident we can achieve the accuracy we desire, but it still unknown
whether we can achieve this without sacrificing our price goals.  This question
is being driven by the potential cost of the optics system.  We may get in touch
with Stephen Fantone again about this.

With more entrants coming to market, it seems that the proper marketization
strategy is no longer simply building a better, cheaper product than the
incumbent.  There are other players who are bringing this value proposition to
the table, and it may be important to take a market position with their entry in mind.

\section{Seven Day Plan}
\begin{itemize}
\item Generate sales estimates and complete the 5 year financial projection
\item Continue progressing on the software development
\item Continue building out the test board
\item Decide whether the next approach at the projection system will be through
  non-coherent light sources with lenses or through lasers and diffusers, and
  order the parts to prototype the system with.
\end{itemize}

\section{People to Meet}

\section{Desired Resources}

\end{document}
