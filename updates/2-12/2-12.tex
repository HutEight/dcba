\documentclass[10pt]{article}

\usepackage{enumitem}
\setlist{nolistsep}

\usepackage[margin=1in]{geometry}

\title{\vspace{-4em}6.S078 Update}
\author{Troy Astorino \and Turner Bohlen \and Craig Cheney \and Gus Downs}
\date{February 12, 2013}
\begin{document}
\maketitle

\section{Plan Progress}
Following the comments we received after being accepted into the class, we
rebroadened our search for potential products, focusing more on market viability
by meeting and interacting with potential customers. We identified five major
areas we wanted to investigate, and after meeting with specialists in those
areas and evaluating potential market sizes, we've narrowed down the scope of
our search. We have two products that we think are very promising, and are
waiting to pass judgment on two other market areas pending a meeting on
Wednesday with a restaurateur and local agriculture proponent and one on
Thursday with the Director of MGH's Innovation and Support Center. We are
meeting with Professors Gifford and Zolot on Friday to discuss our market
analyses and get advice on a decision. We plan on selecting a product and
beginning the design process by Friday.

The products we have narrowed down to:
\begin{itemize}
\item Extensions of the Leap Motion. We see Leap, costing \$70 and accurate to .01
  mm, as a transformational technology for interacting with computers, and think
  there are very strong opportunities to build products extending it. Two
  possible options:
\begin{itemize}
\item Solution to make CAD work practically with the Leap
\begin{itemize}
\item Apparatus to make working with the Leap for extended periods of time
ergonomically practical. At the moment this is impractical due to the need to
wave your hands in the air.
\item Haptic feedback to make using the Leap with CAD models more useful than
  simply using a 3D mouse
\end{itemize}
\item Inexpensive and accurate 3D scanning using the Leap
\begin{itemize}
\item 3D printing is a huge and growing market: Forbes projects it will be a
  \$3 billon industry by 2016. Although this figure includes industrial applications,
  at CES this year consumer 3D printers had a very strong presence. It isn't unrealistic to
  imagine of a near future with a 3D printer in every home. The 3D scanner needs
  to be there too.
\item The high accuracy of the Leap and the low cost makes it an excellent candidate
  to be the basis for a 3D scanning system, potentially by coupling an
  IMU, building a frame, or simply with software to integrate the data.
\end{itemize}
\end{itemize}

\item Inexpensive LIDAR for low-cost robotics. 
\begin{itemize}
\item Low-cost and hobbyist robotics are rapidly growing fields; as measure of size,
  in May 2011 300,000 Arduinos had been sold worldwide
\item LIDAR is extremely useful and common for serious robotic applications, but
  the typical unit price of \$6k makes it inaccessible for consumer or hobbyist
  applications
\end{itemize}
\end{itemize}



\section{Prototype Progress}
As we haven't yet chosen the product we are going to produce, we do not have any
prototype progress this week. 

\end{document}
