\documentclass[8pt]{article}
\newcommand{\margin}{1cm}
\usepackage[colorlinks=true]{hyperref}
\usepackage[top=\margin, bottom=\margin, left=\margin, right=\margin]{geometry}
\usepackage{array}
\usepackage{ragged2e}

\begin{document}

\begin{center}
\LARGE{\textbf{Application for 6.S078 -- Entrepreneurship Project}}
\end{center}

\vspace{.3cm}

\section*{Opportunity Areas}
Our accomplished, enthusiastic team was assembled to have the traits necessary
to build a successful hardware startup. After the past few weeks of
brainstorming and preliminary market research, we have filtered our ideas down
to three products that all fit our skill set and that we would love to create.
We plan to spend the first two weeks of the class continuing to research markets
and interview customers, as well as consult with the course's advisors in order
to choose the product we will develop over the semester.

\subsection*{Potential Projects}
\begin{enumerate}
\item \underline{Specific Molarity Solution Maker} -- Low-cost machine for preparing solutions with arbitrary molarities from stock chemicals. Automates the time-consuming process of calculating and carefully weighing out precise quantities of chemicals for making a solution, along with mixing and autoclaving. Markets include academic and small commercial wet labs.
\item \underline{Robotic Oxygen Tank Carrier} -- Robotic assistant for carrying medical oxygen tanks. The device maintains a specified distance from the user by following a signal generated on the users body. Incorporates a LIDAR system for obstacle detection and path planning. The product would enable greater independence and mobility for individuals who require oxygen tanks. Markets include hospitals, assisted living facilities, nursing homes, and medical rehabilitation centers.
\item \underline{Smart-Home Doorbell} -- Video camera, intercom integrated into doorbell system that allows the homeowner to see and communicate with whoever is at their door via an AV feed to a smartphone app. The user can then choose to unlock the door via the app to let the caller in. Adds convenience and security; would be very useful for package delivery that requires confirmation. Aside from general homeowners, marketable as an inexpensive upgrade to old apartment buzzer systems.
\end{enumerate}


\subsection*{Projected schedule of deliverables}

\begin{tabular}{>{\RaggedLeft}p{6in} | c}
Selected product chosen based on preliminary research of above options & 2/18 \\ \hline
Typical customer and required features based on customer feedback of look-alike models & 3/4 \\ \hline
Work-like model with consolidated feedback from customer reactions & 4/8 \\ \hline
Initial business viability, market size, and manufacturing costs based on work-like and look-like prototypes & 4/15 \\ \hline
Prototype incorporating customer driven design and branding & 4/29 \\ \hline
Draft pitch deck and updated business strategy and market size based on prototype & 5/6 \\ \hline
Final presentation with alpha prototype & 5/13 \\ 

\end{tabular}

\section*{Team Members}

\noindent \textbf{\href{http://www.troyastorino.com}{Troy Astorino}} (AeroAstro,
Physics `13) -- \textbf{About:} Troy's interest in machine learning and robotics
has led him to take classes from a wide variety of departments outside of his
two majors. He participated in StartLabs' C2C program last IAP and firmly
believes that startups can change the world through building profitable
businesses around products and services. \textbf{Skill set:} Troy's experience
with large software systems and his academic focus on probabilistic robotics
will be used in building the software and sensor integration for the project.
\textbf{Desired units:} 24. \textbf{Adviser's email:}
\href{mailto:kwillcox@mit.edu}{kwillcox@mit.edu}.

\vspace{.25cm}

\noindent \textbf{\href{http://www.turnerbohlen.com}{Turner Bohlen}} (Physics,
`14) -- \textbf{About:} While gaining technical knowledge and practice through
classes and internships at startups, Turner has developed an intense interest in
entrepreneurship as a method for bringing novel technology and exceptional
design into the hands of the public. He is the director of StartLabs, a
non-profit student organization dedicated to that mission. \textbf{Skill set:}
Turner has significant experience in software and web development as well as
team management, all of which will be put to use for this project \textbf{Desired units:} 12.
\textbf{Adviser's email:}
\href{mailto:simcoe@space.mit.edu}{simcoe@space.mit.edu}.

\vspace{.25cm}

\noindent \textbf{Craig Cheney} (Mech. E `14) -- \textbf{About:} Craig has
pursued his interest in robotics throughout his career at MIT, most recently
winning the `Intro to Robotics' term project competition, ``Robot Gymnastics''.
\textbf{Skill set:} His expertise in CAD, mechanical design, machining, and
controls \& instrumentation will be utilized in the physical design of the
project, as well as the electronics system. \textbf{Desired units:} 12.
\textbf{Adviser's email:} \href{mailto:ihunter@mit.edu}{ihunter@mit.edu}.

\vspace{.25cm}

\noindent \textbf{Gus Downs} (Physics `13) -- \textbf{About:} Gus has been
heavily involved in experimental physics research across the country during his
time at MIT, designing and building experiments to study ultrafast processes in
quantum materials and efficient cooling of single atoms. \textbf{Skill set:}
Gus's experience with circuit design and signal processing will be put to use
designing the data acquisition and electronics system of the project.
\textbf{Desired units:} 18.  \textbf{Adviser's email:}
\href{mailto:vuletic@mit.edu}{vuletic@mit.edu}.
\end{document}
