\documentclass[10pt]{article}
\usepackage[colorlinks=true]{hyperref}
\usepackage[top=2.3cm, bottom=2.3cm, left=2.3cm, right=2.3cm]{geometry}

\begin{document}

\begin{center}
\LARGE{\textbf{Application for Entrepreneurship Project}}
\end{center}

\section*{Opportunities}

\section*{Projected Schedule}

\subsection*{Deliverables}
\begin{tabular}{l|l}
Present product chosen based on preliminary \\ research of above options & 2/18 \\
Present stereotypical customer and required \\ features based on customer feedback of initial look-alike models & 3/4 \\
Present work-like model with consolidated \\ feedback from customer reactions & 4/8 \\
Initial business strategy incorporating projected \\ manufacturing costs and market
size based on \\ initial prototyping and research & 4/22 \\
Prototype incorporating customer driven \\ design and branding & 4/29 \\
Draft pitch deck & 5/6 \\
Final presentation with alpha prototype & 5/13 \\
\end{tabular}

%\includegraphics{schedule.png}

\section*{Team}
\textbf{\href{http://www.troyastorino.com}{Troy Astorino}} (AeroAstro, Physics `13) -- \textbf{About:} Troy's interest in machine learning and robotics has led him to take classes from a wide variety of departments outside of his two majors. He participated in StartLabs' C2C program last IAP and firmly believes that startups can change the world through building profitable businesses around products and services. \textbf{Skill set:} Troy's experience with large software systems and his academic focus on probabilistic robotics will be used in building the software and sensor integration for the project. 

\vspace{.2cm}

\noindent \textbf{\href{http://www.turnerbohlen.com}{Turner Bohlen}} (Physics, `14) -- \textbf{About:} While gaining technical knowledge and practice through classes and internships at startups, Turner has developed an intense interest in entrepreneurship as a method for bringing novel technology and exceptional design into the hands of the public. He is the director of StartLabs, a non-profit student club dedicated to that mission, and has himself dedicated the last two years to learning as much as possible concerning the process of launching and running a company. \textbf{Skill set:} Turner has significant experience in software and web development that will be put towards writing the software for the project.

\vspace{.2cm}

\noindent \textbf{Craig Cheney} (Mech. E `14) -- \textbf{About:} Craig has pursued his interest in robotics throughout his career at MIT, most recently winning the `Intro to Robotics' term project competition, ``Robot Gymnastics''. \textbf{Skill set:} His expertise in CAD, mechanical design, machining, and controls & instrumentation will be utilized in the physical design of the project, as well as the electronics system.

\vspace{.2cm}

\noindent \textbf{Gus Downs} (Physics `13) -- \textbf{About:} Gus has been heavily involved in experimental physics research across the country during his time at MIT, designing and building experiments to study ultrafast processes in quantum materials and efficient cooling of single atoms. \textbf{Skill set:} Gus's experience with circuit design and signal processing will be put to use designing the data acquisition and electronics system of the project.
\end{document}