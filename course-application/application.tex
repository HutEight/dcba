\documentclass[8pt]{article}
\newcommand{\margin}{1cm}
\usepackage[colorlinks=true]{hyperref}
\usepackage[top=\margin, bottom=\margin, left=\margin, right=\margin]{geometry}
\usepackage{array}
\usepackage{ragged2e}

\begin{document}

\begin{center}
\LARGE{\textbf{Application for Entrepreneurship Project}}
\end{center}

\section*{Opportunity Areas}
Our knowledgeable, excited, and dedicated team was assembled to having the
traits necessary to build a successful hardware startup. After a few weeks of
brainstorming and preliminary market research, we filtered our ideas down to
three products that align with our skill sets and that we'd love to create. We would
like to spend the first two weeks of the class researching markets, interviewing
customers, and consulting with the course advisors in order to choose the
product we will develop over the semester.

\begin{enumerate}
\item Specific Molarity Solution Maker -- Low cost machine for creating
  arbitrary solutions given sufficient input chemicals. Automates the
  high-precision, time-consuming measurement of input quantities along with the
  mixture process. Markets in academic and small commercial wet labs.
\item Oxygen Tank Transportation Platform -- Robotic platform for medical oxygen
  tanks that follows a generated electromagnetic signal. Incorporates a LIDAR
  system for obstacle detection and path planning. Enables greater independence
  and mobility for seniors requiring oxygen tanks. Markets at hospitals,
  assisted living facilities, nursing homes, and medical rehabilitation centers.
\item IP Camera Doorbell -- Video camera integrated into doorbell that allows
  the homeowner to see and communicate with the caller via a video feed to a
  smartphone app and a speaker in the doorbell. The app also can unlocks the
  door. Adds convenience and security; for instance would be very useful for
  package delivery that requires confirmation. Aside from general homeowners,
  marketable as an inexpensive upgrade to old apartment buzzer systems.
\end{enumerate}


\section*{Projected Schedule}

\subsection*{Deliverables}

% \begin{tabular}{l|l}
% Present product chosen based on preliminary \\ research of above options & 2/18 \\
% Present stereotypical customer and required \\ features based on customer feedback of initial look-alike models & 3/4 \\
% Present work-like model with consolidated \\ feedback from customer reactions & 4/8 \\
% Initial business strategy incorporating projected \\ manufacturing costs and market
% size based on \\ initial prototyping and research & 4/22 \\
% Prototype incorporating customer driven \\ design and branding & 4/29 \\
% Draft pitch deck & 5/6 \\
% Final presentation with alpha prototype & 5/13 \\

\begin{tabular}{>{\RaggedLeft}p{6in} | c}
Present product chosen based on preliminary research of above options & 2/18 \\ \hline
Present stereotypical customer and required features based on customer feedback of initial look-alike models & 3/4 \\ \hline
Present work-like model with consolidated feedback from customer reactions & 4/8 \\ \hline
Initial business strategy incorporating projected manufacturing costs and market
size based on initial prototyping and research& 4/22 \\ \hline
Prototype incorporating customer driven design and branding & 4/29 \\ \hline
Draft pitch deck & 5/6 \\ \hline
Final presentation with alpha prototype & 5/13 \\ 

\end{tabular}

%\includegraphics{schedule.png}

\section*{Team}
\textbf{\href{http://www.troyastorino.com}{Troy Astorino}} (AeroAstro, Physics `13) -- \textbf{About:} Troy's interest in machine learning and robotics has led him to take classes from a wide variety of departments outside of his two majors. He participated in StartLabs' C2C program last IAP and firmly believes that startups can change the world through building profitable businesses around products and services. \textbf{Skill set:} Troy's experience with large software systems and his academic focus on probabilistic robotics will be used in building the software and sensor integration for the project. 

\vspace{.2cm}

\noindent \textbf{\href{http://www.turnerbohlen.com}{Turner Bohlen}} (Physics, `14) -- \textbf{About:} While gaining technical knowledge and practice through classes and internships at startups, Turner has developed an intense interest in entrepreneurship as a method for bringing novel technology and exceptional design into the hands of the public. He is the director of StartLabs, a non-profit student club dedicated to that mission, and has himself dedicated the last two years to learning as much as possible concerning the process of launching and running a company. \textbf{Skill set:} Turner has significant experience in software and web development that will be put towards writing the software for the project.

\vspace{.2cm}

\noindent \textbf{Craig Cheney} (Mech. E `14) -- \textbf{About:} Craig has pursued his interest in robotics throughout his career at MIT, most recently winning the `Intro to Robotics' term project competition, ``Robot Gymnastics''. \textbf{Skill set:} His expertise in CAD, mechanical design, machining, and controls \& instrumentation will be utilized in the physical design of the project, as well as the electronics system.

\vspace{.2cm}

\noindent \textbf{Gus Downs} (Physics `13) -- \textbf{About:} Gus has been heavily involved in experimental physics research across the country during his time at MIT, designing and building experiments to study ultrafast processes in quantum materials and efficient cooling of single atoms. \textbf{Skill set:} Gus's experience with circuit design and signal processing will be put to use designing the data acquisition and electronics system of the project.
\end{document}