\documentclass{article}

\usepackage{titlesec}
\newcommand{\sectionbreak}{\clearpage}

\begin{document}
\section{Executive Summary}
3D scanning has a wide variety of applications, but the high cost of these
scanners keeps the technology from being used in many potential applications.
Bringing down the cost and complexity of accurate 3D scanning would enable the
use of 3D scanning in many applications where it now is prohibitively expensive:
research labs, hobbyist builders, 3D printing enthusiasts, artists, independent
product designers, small manufacturers, etc.

We found that there is an academically popular method, structured light
scanning, that maintains high scan accuracy, at lower costs. However, there is
typically a large degree of complexity in setting up a structured light scanner;
streamlining this service and packaging it into a plug-and-play device will
enable us to sell the product to markets that currently cannot access or afford
3D scanning. Additionally, we have ways to make the product even less expensive
than typical structured light scanners by substituting inexpensive components,
such as smart phone cameras, in clever ways.

In short, we want to bring 3D scanning to a large group of users who cannot
access it today. We will do this by building a low-cost, user-friendly, and
high-accuracy, 3D scanner.

\section{Market Overview}
Our competitors can be separated into 3 categories:

The current lowest-cost high-accuracy options: The major competitors in this
range would be the NextEngine scanner or Artec's 3D scanners. Our price (~\$500)
will be significantly lower than these products, and our scanner will be easier
to use than the NextEnginer scanner. The NextEngine costs over \$3,000 (once
you've purchased all the necessary software), and the least expensive Artec
scanner costs \$12,000. Our scanner's accuracy will be in the same range as
these products. Unlike these products, however, our scanner will only be able to
scan objects that are under a certain size. While the Artec scanners make
scanning large objects fairly simple, scanning large objects with the NextEngine
is very inconvenient, which is also true for other products in the their range.

The very low-cost (free) options: There are some options on the market that come
in at a price lower than ours. AutoDesk as an free app called 123D Catch--you
take a series of images of an object and then it performs image reconstruction
to generate a 3D model. Some products (like Scanect) allow a customer to use a
Kinect to perform 3D scanning, and Microsoft recently released KinectFusion,
which turns the Kinect into a 3D scanner for anyone with the Windows Kinect SDK.
These technologies, though, do not provide the level of accuracy we are seeking,
or the ease of access. They have more difficulty with highly specular objects
than we will, and require user participation during the entire scanning process.

The emerging competitors: There are a couple competitors that have announced
products but don't have anything for sale yet. CADScan successfully completed a
Kickstarter project for a desktop 3D scanner that will be similar to ours in
terms of target specifications and customers. From the Kickstarter funding
levels, it seems that they plan on charging at least £650 (\$1000) for the
scanner, a price we are planning to come in significantly below. MakerBot also
just announced at SXSW that they will be producing a desktop 3D scanner called
the Digitizer. The announcement was very nebulous on details, but given their 3D
printer costs and their description of the Digitizer, we surmise that they will
be charging over \$1500.

We are targeting individual consumers and small enterprises.

The individual consumers we are targeting are hobbyists, enthusiasts, and
artists. Many people have become excited about desktop 3D printing, and will be
similarly excited by desktop 3D scanning. There is evidence for this in
CADScan's successful Kickstarter campaign, and the excitement surrounding
MakerBot's SXSW Digitizer announcement. Though much of the popular focus has
been on scanning objects in order to get a model to print, other uses of 3D
scanners can be marketed. A sculptor may want to display his work online, or
send an early version of a sculpture to a colleague; our scanner allows him to
do so with ease.


The small enterprises we are focusing on include research labs, product
designers, and small manufacturers. In research labs, it is often important to
have models and measurements of objects being tested; while the 3D scanners
available today would be an unaffordable luxury, ours would be low-cost enough
to have a practical purpose. Product designers can make initial designs with
physical materials, which are nicer to work with, and then scan them to get a
usable CAD model. Small manufacturers and machine shops can use them in the same
fashion that large manufacturers do: for automated verification of manufactured
parts, as well as for designing parts with unknown dimensions.

\section{Innovations}

\section{Lessons Learned}

\section{Plan of Action}
What do you want to achieve this summer? We will work with you to create
rigorous yet achievable milestones, but we'd first like to hear from you. Where
do you want your team to be by mid-September regarding customers, product, team,
and finances? For more explanation about milestones, including examples, please
go to http://entrepreneurship.mit.edu/fsa *Proposed Customer Milestones (list
2-5) 1) Decision on whether we can successfully market to both hobbyist
consumers and small businesses/labs. 2) (Possibly) Launch of Kickstarter
campaign to raise excitement and additional funding/gauge interest OR secure a
list of beta testers to launch a private beta of the product. 3) Choose product
name as a reflection of desired brand image

*Proposed Product Milestones (list 2-5) 1) Finalize which low-cost projection
method will be used in the scanner. 2) Complete an alpha prototype of the
scanner. 3) Create a component list for a beta version of the product 4) Develop
an IP strategy to protect the design components of the product. Determine
whether filing for provisional patents is appropriate (most likely for the
projection technique). If it is, file the patents.

*Proposed Team Milestones (list 2-5) 1) Clearly define team members' roles on
the team, exploiting strengths and weaknesses. 2) Finalize team members' equity
splits and vesting schedules, defining what will happen under different possible
scenarios. Specifically, what will happen if a team member wants finish school,
working on the startup on a part-time basis, and then returns full-time to the
company. 3) Identify gaps on the team, develop a plan to fill them, and execute
it. Specifically, plan a path to cope with a team member returning to school at
the end of the summer.

*Proposed Financial Milestones (list 2-5) 1) Identify projected unit
manufacturing cost at various build volumes, using beta prototype component
list. 2) Develop a high quality short and long pitch deck for financial
presentations. 3) Apply to VCs, angels, or further accelerators as appropriate
to secure funding for after the summer.

\section{Risk Factors}

\section{Team}
*Troy's Bio Troy is graduating this June with majors in 16-ENG (concentration in
Robotics) and 8B (concentration in Computational Learning Systems), and a minor
in 14. The majority of his experience is as a programmer, focusing on machine
learning and web development. He plays tennis for MIT and is a Freshman
Leadership Program counselor.

*Craig's Bio Craig is pursuing his degree in 2A-6, Mechanical Engineering with a
concentration in Control, Instrumentation and Robotics. He has worked
extensively with robotics and machine design. Craig is currently president of
his fraternity, captain of the varsity Waterpolo team, and a member of the
national varsity swim team.

*Gus's Bio Gus is receiving his BS in physics from MIT this June. His focus has
been on ultra-cold atomic physics, and he has worked in 5 different labs ranging
from condensed matter physics to plasma physics. He sings with the Logarhythms,
is a Freshman Leadership Program counselor, is a (half) Iron Man, and cooks a
mean pork roulade.

* How long have you known each other, and what have you worked on in the past?
(Include past work done on this project, if applicable.) We have been working on
this project together since January, as part of 6.S078, Entrepreneurship
Project. We have been working with another student as well, Turner Bohlen.
Unfortunately Turner won't be able to join us for the summer as he has previous
commitments.

Troy and Gus have known each other since their freshman year, and have been
co-counselors in the Freshman Leadership Program for the past 2 years. Craig and
Troy took the same robotics class, 2.12 Intro to Robotics, in the Fall of 2012
where they both placed in the top 5 out of 70 students.

\section{Financial Plan}

\end{document}
